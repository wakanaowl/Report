\documentclass[dvipdfmx,autodetect-engine,titlepage]{jsarticle}
\usepackage[dvipdfm]{graphicx}
\usepackage{ascmac}
\usepackage{fancybox}
\usepackage{listings}
\usepackage{plistings}
\usepackage{itembkbx}
\usepackage{amsmath}
\usepackage{url}
\usepackage{graphics}
\usepackage{listings}
\usepackage{here}

\lstset{%
  language={C},
  basicstyle={\small},%
  identifierstyle={\small},%
  commentstyle={\small\itshape\color[rgb]{0,0.5,0}},%
  keywordstyle={\small\bfseries\color[rgb]{0,0,1}},%
  ndkeywordstyle={\small},%
  stringstyle={\small\ttfamily\color[rgb]{1,0,1}},
  frame={tb},
  breaklines=true,
  columns=[l]{fullflexible},%
  numbers=left,%
  xrightmargin=0zw,%
  xleftmargin=3zw,%
  numberstyle={\scriptsize},%
  stepnumber=1,
  numbersep=1zw,%
  lineskip=-0.5ex%
}

\textheight=23cm
\renewcommand{\figurename}{図}
\renewcommand{\tablename}{表}
\newenvironment{code}
{\vspace{0.5zw}\VerbatimEnvironment  \begin{screen} 
\baselineskip=1.0\normalbaselineskip
 \begin{Verbatim}}
{\end{Verbatim}
\baselineskip=\normalbaselineskip
 \end{screen}\vspace{0.5zw}} 

\title{ワイヤレス通信システム(B1)\\
4th Week 磁気ダイポール\\
}
\author{2600200087-2\\Oku Wakana\\奥 若菜}
\date{May. 29 2022}

\begin{document}

\maketitle

\section{磁気ダイポールによる電界}
教科書の図2.3のような磁気ダイポールによる電界を表す式(2・24)を導出する。\\
はじめに、微小ループ電流による磁界は、ループ面に垂直な微小磁気ダイポールと等価なので、図2.2のような電気ダイポールの電磁界から求める。\\

ファラデーの法則、アンペア・マクスウェルの法則はそれぞれ、
\begin{eqnarray}
  \nabla \times E &=& - \frac{\partial B}{\partial t}\\
  \nabla \times H &=& J + \frac{\partial D}{\partial t}
\end{eqnarray}
また、ベクトルポテンシャルAは、自由空間において
\begin{eqnarray}
  B &=& \nabla \times A 
\end{eqnarray}
式(1)に式(3) を代入すると
\begin{eqnarray}
  \nabla \times E = -\frac{\partial}{\partial t}(\nabla \times A)\nonumber\\
  \nabla \times (E+\frac{\partial A}{\partial t})= 0\nonumber
\end{eqnarray}




\end{document}
