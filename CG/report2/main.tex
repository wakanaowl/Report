\documentclass[dvipdfmx,autodetect-engine,titlepage]{jsarticle}
\usepackage[dvipdfm]{graphicx}
\usepackage{ascmac}
\usepackage{fancybox}
\usepackage{listings}
\usepackage{plistings}
\usepackage{itembkbx}
\usepackage{amsmath}
\usepackage{url}
\usepackage{graphics}
\usepackage{listings}
\usepackage{here}

\lstset{%
  language={C},
  basicstyle={\small},%
  identifierstyle={\small},%
  commentstyle={\small\itshape\color[rgb]{0,0.5,0}},%
  keywordstyle={\small\bfseries\color[rgb]{0,0,1}},%
  ndkeywordstyle={\small},%
  stringstyle={\small\ttfamily\color[rgb]{1,0,1}},
  frame={tb},
  breaklines=true,
  columns=[l]{fullflexible},%
  numbers=left,%
  xrightmargin=0zw,%
  xleftmargin=3zw,%
  numberstyle={\scriptsize},%
  stepnumber=1,
  numbersep=1zw,%
  lineskip=-0.5ex%
}

\textheight=23cm
\renewcommand{\figurename}{図}
\renewcommand{\tablename}{表}
\newenvironment{code}
{\vspace{0.5zw}\VerbatimEnvironment  \begin{screen} 
\baselineskip=1.0\normalbaselineskip
 \begin{Verbatim}}
{\end{Verbatim}
\baselineskip=\normalbaselineskip
 \end{screen}\vspace{0.5zw}} 

\title{コンピュータグラフィックス(Q)\\
ミニレポート(1)\\
物体をシーンに配置する(合成変換)\\
}
\author{2600200087-2\\Oku Wakana\\奥 若菜}
\date{Nov. 18 2022}

\begin{document}

\maketitle

\section*{問題}
\subsection*{考えてみよう(4)}
点(1,0,0)を原点を中心に2倍拡大し、z軸方向に+10並行移動させる。\\
\begin{eqnarray*}
  \begin{pmatrix}
    2 & 0 & 0 & 0 \\
    0 & 2 & 0 & 0 \\
    0 & 0 & 2 & 10 \\
    0 & 0 & 0 & 1 \\
  \end{pmatrix} 
  \begin{pmatrix}
    1 \\ 
    0 \\ 
    0 \\ 
    1 
  \end{pmatrix}
  = \begin{pmatrix}
    2 \\ 
    0 \\ 
    10 \\ 
    1 \\
  \end{pmatrix}
\end{eqnarray*}
 \\
よって、点(1,0,0)は点(2,0,10)に変換される。\\


\subsection*{考えてみよう(5)}
点(1,0,0)をz軸方向に+10並行移動させ、原点を中心に2倍拡大する。\\
\begin{eqnarray*}
  \begin{pmatrix}
    2 & 0 & 0 & 0 \\
    0 & 2 & 0 & 0 \\
    0 & 0 & 2 & 20 \\
    0 & 0 & 0 & 1 \\
  \end{pmatrix} 
  \begin{pmatrix}
    1 \\ 
    0 \\ 
    0 \\ 
    1 
  \end{pmatrix}
  = \begin{pmatrix}
    2 \\ 
    0 \\ 
    20 \\ 
    1 \\
  \end{pmatrix}
\end{eqnarray*}
 \\
よって、点(1,0,0)は点(2,0,20)に変換される。

\end{document}
