\documentclass[dvipdfmx,autodetect-engine,titlepage]{jsarticle}
\usepackage[dvipdfm]{graphicx}
\usepackage{ascmac}
\usepackage{fancybox}
\usepackage{listings}
\usepackage{plistings}
\usepackage{itembkbx}
\usepackage{amsmath}
\usepackage{url}
\usepackage{graphics}
\usepackage{listings}
\usepackage{here}

\lstset{%
  language={C},
  basicstyle={\small},%
  identifierstyle={\small},%
  commentstyle={\small\itshape\color[rgb]{0,0.5,0}},%
  keywordstyle={\small\bfseries\color[rgb]{0,0,1}},%
  ndkeywordstyle={\small},%
  stringstyle={\small\ttfamily\color[rgb]{1,0,1}},
  frame={tb},
  breaklines=true,
  columns=[l]{fullflexible},%
  numbers=left,%
  xrightmargin=0zw,%
  xleftmargin=3zw,%
  numberstyle={\scriptsize},%
  stepnumber=1,
  numbersep=1zw,%
  lineskip=-0.5ex%
}

\textheight=23cm
\renewcommand{\figurename}{図}
\renewcommand{\tablename}{表}
\newenvironment{code}
{\vspace{0.5zw}\VerbatimEnvironment  \begin{screen} 
\baselineskip=1.0\normalbaselineskip
 \begin{Verbatim}}
{\end{Verbatim}
\baselineskip=\normalbaselineskip
 \end{screen}\vspace{0.5zw}} 

\title{情報アクセス論(後半)レポート\\
}
\author{セキュリティ・ネットワークコース\\2600200087-2\\Oku Wakana\\奥 若菜}
\date{Jun.14 2023}

\begin{document}

\maketitle

\section{協調フィルタリング}
購入者Cの商品4に対する評価値を,協調フィルタリングの方法を用いて算出せよ.
さらに購入者Cに商品4は推薦すべきか、推薦すべきでないかも回答せよ.\\

まず,購入者間の類似度を,購入者が付与した評価値より計算する.
利用者間の類似度は相関係数により算出する.購入者Cと購入者Aの相関係数,購入者Cと購入者Bの相関係数をそれぞれ求める.また,
\begin{math}
  \overline{A}, \overline{B}, \overline{C}
\end{math}
は各購入者の評価値の平均とし,
\begin{math}
  \sqrt{3} = 1.73, \sqrt{7} = 2.65 
\end{math}
とする.

\begin{eqnarray*}
  r_{C,A} &=& \frac{\sum_{i} (C_{i} - \overline{C})(A_{i} - \overline{A}) }{\sqrt{\sum_{i} (C_{i} - \overline{C})^2} \sqrt{\sum_{i} (A_{i} - \overline{A})^2}} \\\\
  &=& \frac{(2-2)(4-\frac{11}{3} ) + (1-2)(5-\frac{11}{3} ) + (3-2)(2-\frac{11}{3} )}{\sqrt{(2-2)^2+(1-2)^2+(3-2)^2} \sqrt{(4-\frac{11}{3} )^2+ (5-\frac{11}{3} )^2 + (2-\frac{11}{3} )^2}} \\\\
  &=& - \frac{9}{2\sqrt{21}  } \\\\
  &=& -0.98
\end{eqnarray*}

\begin{eqnarray*}
  r_{C,B} &=& \frac{\sum_{i} (C_{i} - \overline{C})(B_{i} - \overline{B}) }{\sqrt{\sum_{i} (C_{i} - \overline{C})^2} \sqrt{\sum_{i} (B_{i} - \overline{B})^2}} \\\\
  &=& \frac{(2-2)(1-3) + (1-2)(3-3) + (3-2)(5-3)}{\sqrt{(2-2)^2+(1-2)^2+(3-2)^2} \sqrt{(1-3)^2+ (3-3)^2 + (5-3)^2}} \\\\
  &=&  \frac{1}{2} \\\\
  &=& 0.50 \\
\end{eqnarray*}

相関係数より,購入者Cと購入者Aの嗜好は正反対であり,利用者Cと利用者Bの嗜好は少し似ていることが分かる.
続いて,購入者Cの商品4に対する評価値を算出する.

\begin{eqnarray*}
  predict(C,4) &=& \overline{C} + \frac{\sum_{J}\epsilon_{User}(J_{4}-\overline{J})r_{C,J}}{\sum_{J}\epsilon_{User}| r_{C,J}\vert } \\\\
  &=& 2 + \frac{(1-3)(-0.98)+(4-2.60 )0.50}{0.98 + 0.50} \\\\
  &=& 4.21 \\
\end{eqnarray*}

購入者Cの商品4に対する評価値は4.21となった.これは利用者Cの評価値の平均の3よりも高い.
したがって,購入者Cに商品4は推薦すべきである.



\section{単語の共起関係の評価}
任意の単語のペア1,2 に対し,どちらの方が関連の強い単語のペアであるかを評価する.
単語「立命館大学」と関連が強いと考えられる単語を「西園寺」とし.関連が弱いと考えられる単語を「寿司」とする.\\\\
単語のペア1:立命館大学 西園寺 \\
単語のペア2:立命館大学 寿司 \\\\
検索エンジンにGoogleを用いて,各単語と,単語のペアの検索ヒット数を調べた.結果を表1に示す.

\begin{table}[H]
  \centering
  \caption{各単語の検索ヒット数}
  \label{table: Artifact}
  \begin{tabular}{|l|l|}
  \hline
  単語,単語のペア  & 検索ヒット数      \\ \hline
  立命館大学     & 15,600,000  \\ \hline
  西園寺       & 9,920,000   \\ \hline
  寿司        & 428,000,000 \\ \hline
  立命館大学 西園寺 & 42,200      \\ \hline
  立命館大学 寿司  & 474,000     \\ \hline
  \end{tabular}
  \end{table}
 \\
表1に示した検索ヒット数を用い,単語ペア1と2の相互情報量を算出する.ここでは,検索エンジンの総Webページ数を1兆と仮定する.

\begin{eqnarray*}
  I(立命館大学,西園寺) &=& log\frac{P(立命館大学,西園寺)}{P(立命館大学) P(西園寺)} \\\\
  &=& \frac{\frac{42,200}{1,000,000,000,000} }{\frac{15,600,000}{1,000,000,000,000} \times  \frac{9,920,000}{1,000,000,000,000} } \\\\
  &=& 272.694\dots
\end{eqnarray*}

\begin{eqnarray*}
  I(立命館大学,寿司) &=& log\frac{P(立命館大学,寿司)}{P(立命館大学) P(寿司)} \\\\
  &=& \frac{\frac{474,000}{1,000,000,000,000} }{\frac{15,600,000}{1,000,000,000,000} \times  \frac{428,000,000}{1,000,000,000,000} } \\\\
  &=& 70.992\dots \\
\end{eqnarray*}

結果は,単語ペア1の相互情報量のほうが高くなった.
このことから,単語「立命館大学」は,単語「寿司」より,単語「西園寺」のほうが共有する情報量が多いと分かる.
よって,予想と同じ結果になったと言える.

\section{単純ベイズ分類器による分類}
(1)spam クラスに単語 banking が存在する条件付き確率
\begin{eqnarray*}
  P(banking | spam) = \frac{2}{3} \\
\end{eqnarray*}

(2)not spam クラスに単語 buy が存在する条件付き確率
\begin{eqnarray*}
  P(buy | not spam) = \frac{2}{7} \\
\end{eqnarray*}

(3)11番目のメールがspamである条件付き確率\\\\
11 番目のメールが与えられ,以下のように単語ベクトルで表現できるとする. 
\begin{eqnarray*}
  mail\_11 = (1, 1, 1, 0, 1)\\
\end{eqnarray*}


文書mail\_11が分類されるクラスcを,
\begin{math}
  Class(mail\_11)
\end{math}
で表すと,
\begin{math}
  Class(mail\_11)
\end{math}
は\begin{math}
  P(c| mail\_11)
\end{math}
が最大となるクラスcとなる.分類されるクラスの集合をCとして,これを表現すると,
\begin{eqnarray*}
  Class(mail\_11) = arg_{c\epsilon C} maxP(c| mail\_11) = arg_{c\epsilon C} maxP(mail\_11| c) P(c) \\
\end{eqnarray*}


% まず,spamクラスを観測したとき,文書mail\_11である条件付き確率\begin{math}
%   P(mail\_11| spam)
% \end{math}を求める.

spamクラスを観測する確率P(spam)は,
\begin{eqnarray*}
  P(spam) = \frac{3}{10} \\
\end{eqnarray*}

spamクラスを観測したとき,文書mail\_11である条件付き確率は以下の式で表現できる.ただし,Wは文書集合中の全ての単語の
集合を表し,\begin{math}
  \delta(w,mail\_11)
\end{math}
は単語wが文書mail\_11に出現したときに1,それ以外のときに0を返す関数とする.
\begin{eqnarray*}
  P(mail\_11| spam) = \prod_{w\epsilon W} P(w| spam)^{\delta(w,mail\_11)} (1-P(w|spam))^{\delta(w,mail\_11)} \\
\end{eqnarray*}

spamクラスに属する文書が,単語wを含む条件付き確率
\begin{math}
  P(w|spam)
\end{math}
をそれぞれ求めると,
\begin{eqnarray*}
  P(cheap|spam) = 1 \\
  P(buy|spam) = \frac{1}{2}  \\
  P(banking|spam) = \frac{3}{4}  \\
  P(dinner|spam) = \frac{1}{4}  \\
  P(the|spam) = 1  \\
\end{eqnarray*}

spamクラスに属する全ての文書が特定の単語wを含むことで,条件付き確率
\begin{math}
  P(w|spam)
\end{math}が1になる場合がある.これにより
\begin{math}
  1-P(w|spam) 
\end{math}
が0になり,\begin{math}
  P(mail\_11|spam)
\end{math}が求められなくなる.この問題を回避するために,分母に2,分子に1を加えるラプラススムージングを適用する.\\\\
ラプラススムージングを適用した場合の条件付き確率\begin{math}
  P(w|spam)
\end{math}
\begin{eqnarray*}
  P(cheap|spam) = \frac{4}{5} \\
  P(buy|spam) = \frac{2}{5}  \\
  P(banking|spam) = \frac{3}{5}  \\
  P(dinner|spam) =  \frac{1}{5} \\
  P(the|spam) = \frac{4}{5} \\
\end{eqnarray*}

よって,文書mail\_11がspamである条件付き確率\begin{math}
  P(mail\_11|spam)
\end{math}は,
\begin{eqnarray*}
  P(mail\_11| spam) &=& \frac{4}{5}(1-\frac{4}{5}) \cdot \frac{2}{5}(1-\frac{2}{5})\cdot \frac{3}{5}(1-\frac{3}{5})\cdot \frac{1}{5}^0(1-\frac{1}{5})^0\cdot \frac{4}{5} (1-\frac{4}{5} ) \\\\
  &=& \frac{576}{5^8} \\\\
  &=& 0.00147 \dots \\
\end{eqnarray*}

% 同様に,not spamクラスを観測したとき,文書mail\_11である条件付き確率\begin{math}
%   P(mail\_11|notspam)
% \end{math}
% を求める.\\
% not spamクラスに属する文書が,単語wを含む条件付き確率
% \begin{math}
%   P(w|notspam)
% \end{math}
% \begin{eqnarray*}
%   P(cheap|spam) = \frac{2}{9} \\
%   P(buy|spam) = \frac{3}{9}  \\
%   P(banking|spam) = \frac{3}{9}  \\
%   P(dinner|spam) =  \frac{2}{9} \\
%   P(the|spam) = \frac{8}{9} \\
% \end{eqnarray*}

% よって,not spamクラスを観測したとき,文書mail\_11である条件付き確率\begin{math}
%   P(mail\_11|notspam)
% \end{math}は,
% \begin{eqnarray*}
%   P(mail\_11| spam) &=& \frac{2}{9}(1-\frac{2}{9}) \cdot \frac{3}{9}(1-\frac{3}{9})\cdot \frac{3}{9}(1-\frac{3}{9})\cdot \frac{2}{9}^0(1-\frac{2}{9})^0\cdot \frac{8}{9} (1-\frac{8}{9} ) \\\\
%   &=& \frac{36288}{9^8} \\\\
%   &=& 0.00084 \dots \\
% \end{eqnarray*}

% 以上より,
% クラスcはspamであるとき\begin{math}
%   P(c| mail\_11)
% \end{math}
% が最大となることから,mail\_11はspamだ!!!!


\end{document}
