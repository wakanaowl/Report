\documentclass[dvipdfmx,autodetect-engine,titlepage]{jsarticle}
\usepackage[dvipdfm]{graphicx}
\usepackage{ascmac}
\usepackage{fancybox}
\usepackage{listings}
\usepackage{plistings}
\usepackage{itembkbx}
\usepackage{amsmath}
\usepackage{url}
\usepackage{graphics}
\usepackage{listings}
\usepackage{here}

\lstset{%
  language={C},
  basicstyle={\small},%
  identifierstyle={\small},%
  commentstyle={\small\itshape\color[rgb]{0,0.5,0}},%
  keywordstyle={\small\bfseries\color[rgb]{0,0,1}},%
  ndkeywordstyle={\small},%
  stringstyle={\small\ttfamily\color[rgb]{1,0,1}},
  frame={tb},
  breaklines=true,
  columns=[l]{fullflexible},%
  numbers=left,%
  xrightmargin=0zw,%
  xleftmargin=3zw,%
  numberstyle={\scriptsize},%
  stepnumber=1,
  numbersep=1zw,%
  lineskip=-0.5ex%
}

\textheight=23cm
\renewcommand{\figurename}{図}
\renewcommand{\tablename}{表}
\newenvironment{code}
{\vspace{0.5zw}\VerbatimEnvironment  \begin{screen} 
\baselineskip=1.0\normalbaselineskip
 \begin{Verbatim}}
{\end{Verbatim}
\baselineskip=\normalbaselineskip
 \end{screen}\vspace{0.5zw}} 

\title{コンピュータグラフィックス(Q)\\
ミニレポート(2)\\
投影と画像生成\\
}
\author{2600200087-2\\Oku Wakana\\奥 若菜}
\date{Oct.29 2022}

\begin{document}

\maketitle

\section*{問題}
視点座標から2次元スクリーン座標へのx座標の変
換式で,簡単のため,d=1とおいて,以下とする
\begin{eqnarray*}
  x_{S} = \frac{1}{z_{E}}x_{E} \\\\
  y_{S} = \frac{1}{z_{E}}y_{E} \\
\end{eqnarray*}
点\begin{math}
  A(1,0,z)_{E} と B(2,0,z)_{E}
\end{math}
のスクリーン座標での距離を,Z=2,4,6,8,10の場合についてそれぞれ求める.\\\\

表にまとめたものが以下である.\\

\begin{table}[H]
  \centering
  \begin{tabular}{|c|c|c|c|}
  \hline
  z  & A(xs,ys) & B(xs,ys) & AB間の距離 \\ \hline
  2  & (1/2,0)  & (1,0)    & 1/2    \\ \hline
  4  & (1/4,0)  & (1/2,0)  & 1/4    \\ \hline
  6  & (1/6,0)  & (1/3,0)  & 1/6    \\ \hline
  8  & (1/8,0)  & (1/4,0)  & 1/8    \\ \hline
  10 & (1/10,0) & (1/5,0)  & 1/10   \\ \hline
  \end{tabular}
  \end{table}


\end{document}
