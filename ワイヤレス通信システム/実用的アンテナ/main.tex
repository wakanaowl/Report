\documentclass[dvipdfmx,autodetect-engine,titlepage]{jsarticle}
\usepackage[dvipdfm]{graphicx}
\usepackage{ascmac}
\usepackage{fancybox}
\usepackage{listings}
\usepackage{plistings}
\usepackage{itembkbx}
\usepackage{amsmath}
\usepackage{url}
\usepackage{graphics}
\usepackage{here}

\lstset{
  basicstyle={\ttfamily},
  identifierstyle={\small},
  commentstyle={\smallitshape},
  keywordstyle={\small\bfseries},
  ndkeywordstyle={\small},
  stringstyle={\small\ttfamily},
  frame={tb},
  breaklines=true,
  columns=[l]{fullflexible},
  numbers=left,
  xrightmargin=0zw,
  xleftmargin=3zw,
  numberstyle={\scriptsize},
  stepnumber=1,
  numbersep=1zw,
  lineskip=-0.5ex
}

\textheight=23cm
\renewcommand{\figurename}{図}
\renewcommand{\tablename}{表}
\newenvironment{code}
{\vspace{0.5zw}\VerbatimEnvironment  \begin{screen} 
\baselineskip=1.0\normalbaselineskip
 \begin{Verbatim}}
{\end{Verbatim}
\baselineskip=\normalbaselineskip
 \end{screen}\vspace{0.5zw}} 

\title{ワイヤレス通信システム(B1)\\
11th Week 実用的アンテナ\\
}
\author{2600200087-2\\Oku Wakana\\奥 若菜}
\date{July 10 2022}

\begin{document}

\maketitle
\section*{演習問題}
\subsection*{問2}
放射器\#1,導波器\#2,反射器\#3から構成される3素子八木・宇田アンテナの入力インピーダンスを\begin{math}
  Z_{11}からZ_{33}
\end{math}で表記される自己および相互インピーダンスを用いて導出せよ。

\subsubsection*{解答}
放射器\#1の給電点にかかる電圧は、
\begin{eqnarray}
  V_1 = Z_{11}I_1 + Z_{12}I_2 + Z_{13}I_3
\end{eqnarray}
 \\
また、導波器\#2,反射器\#3は電圧がかからないため、

\begin{eqnarray}
  V_2 = Z_{21}I_1 + Z_{22}I_2 + Z_{23}I_3 = 0\\\nonumber\\
  V_3 = Z_{31}I_1 + Z_{32}I_2 + Z_{33}I_3 = 0
\end{eqnarray}
 \\
(1)〜(3)より入力インピーダンス\begin{math}
  Z_{in}
\end{math}は、

\begin{eqnarray*}
  Z_{in} = Z_{11} + \frac{2Z_{12}Z_{23}Z_{31} - Z_{22}{Z_{13}}^2 - Z_{33}{Z_{12}}^2}{Z_{22}Z_{33}-{Z_{23}}^2}
\end{eqnarray*}

 \\
\subsection*{問3}
直径9mmの十分薄く電気・磁気的に影響のないコアに巻かれたノーマルモードヘリカルアンテナが周波数1100MHzにおいて円偏波を放射する条件を示せ。

\subsubsection*{解答}

周波数1100MHzにおける波長λは、
\begin{eqnarray*}
  \lambda  = \frac{3.0 \times 10^8}{1.1 \times 10^9} = 0.273
\end{eqnarray*}
 \\
\begin{math}
  E_{\theta }とE_{\phi }
\end{math}は位相が直交しているため、円偏波の条件は
\begin{math}
  | E_{\theta } \vert = | E_{\phi }\vert 
\end{math}である。\\

ヘリカルの1周期の周長は、
\begin{eqnarray*}
  C = 2\pi a = 
\end{eqnarray*}

\end{document}
