\documentclass[dvipdfmx,autodetect-engine,titlepage]{jsarticle}
\usepackage[dvipdfm]{graphicx}
\usepackage{ascmac}
\usepackage{fancybox}
\usepackage{listings}
\usepackage{plistings}
\usepackage{itembkbx}
\usepackage{amsmath}
\usepackage{url}
\usepackage{graphics}
\usepackage{listings}
\usepackage{here}

\lstset{%
  language={C},
  basicstyle={\small},%
  identifierstyle={\small},%
  commentstyle={\small\itshape\color[rgb]{0,0.5,0}},%
  keywordstyle={\small\bfseries\color[rgb]{0,0,1}},%
  ndkeywordstyle={\small},%
  stringstyle={\small\ttfamily\color[rgb]{1,0,1}},
  frame={tb},
  breaklines=true,
  columns=[l]{fullflexible},%
  numbers=left,%
  xrightmargin=0zw,%
  xleftmargin=3zw,%
  numberstyle={\scriptsize},%
  stepnumber=1,
  numbersep=1zw,%
  lineskip=-0.5ex%
}

\textheight=23cm
\renewcommand{\figurename}{図}
\renewcommand{\tablename}{表}
\newenvironment{code}
{\vspace{0.5zw}\VerbatimEnvironment  \begin{screen} 
\baselineskip=1.0\normalbaselineskip
 \begin{Verbatim}}
{\end{Verbatim}
\baselineskip=\normalbaselineskip
 \end{screen}\vspace{0.5zw}} 

\title{情報アクセス論(前半)レポート\\
}
\author{セキュリティ・ネットワークコース\\2600200087-2\\Oku Wakana\\奥 若菜}
\date{Jun.14 2023}

\begin{document}

\maketitle

\section{問1}
以下の索引語-文書行列Dから,問合せqと文書d1〜d3それぞれのコサイン類似度を求めよ.
\begin{eqnarray}
  D =
  \begin{bmatrix}
    0 & 2 & 2\\
    3 & 2 & 1\\
    3 & 0 & 2\\
  \end{bmatrix}
\end{eqnarray}

\begin{eqnarray}
  q =
  \begin{bmatrix}
    1 \\
    0 \\
    1 \\
  \end{bmatrix}
\end{eqnarray}

\subsection{d1}

\begin{eqnarray}
  \cos (d_{1},q) = \frac{0\cdot 1 + 3\cdot 0 + 3\cdot 1}{\sqrt{0^2+0^2+3^2} \sqrt{1^2+0^2+1^2} } 
  = \frac{1}{\sqrt{2} } 
  = 0.71
\end{eqnarray}

\subsection{d2}

\begin{eqnarray}
  \cos (d_{2},q) = \frac{2\cdot 1 + 2\cdot 0 + 0\cdot 1}{\sqrt{2^2+2^2+0^2} \sqrt{1^2+0^2+1^2} } 
  = \frac{1}{2} 
  = 0.50
\end{eqnarray}

\subsection{d3}

\begin{eqnarray}
  \cos (d_{3},q) = \frac{2\cdot 1 + 1\cdot 0 + 2\cdot 1}{\sqrt{2^2+1^2+2^2} \sqrt{1^2+0^2+1^2} } 
  = \frac{4}{3\sqrt{2} } 
  = 0.95
\end{eqnarray}
 \\


\subsection{考察}
問合せqと文書d1〜d3それぞれのコサイン類似度を求めた結果,文書d3が最も類似度が高いと分かった.各索引語の出現頻度を


\section{問2}
問合せ「アップル」を用いてWeb検索エンジンGoogleで検索し,検索結果の上位5件について,順位とタイトルを記載し,個々の検索結果について検索の有効性(適合性,適切性,有用性)の観点から考察せよ.\\

検索結果を以下に示す.
\begin{enumerate}
  \item Apple - 公式サイト
  \item Apple(日本)
  \item Apple - Wikipedia
  \item 中古車買取・中古車査定のアップル【公式】
  \item アップル 株価[AAPL/Apple]最新ニュース 日経会社情報DIGITAL
\end{enumerate}
 \\
検索者は「Apple Inc.(米国のテクノロジー企業)についてさまざまな情報を知りたい」という情報要求を持っている.
検索結果の上位5件のうち,4番目の『中古車買取・中古車査定のアップル【公式】』除く4つは全てApple Inc.に関する文書であるため,これらの文書は適合性と適切性を満たしていると言える.\\
一方で『中古車買取・中古車査定のアップル【公式】』は,Apple Inc.に関する情報を知りたかった検索者にとって不要なものであるため,適切性を満たしていないと言える.ただし,「アップル」
という問合せに対してアップルという名称の別の企業が検索されたという事実は,客観的にみて誤りではない.したがって,適合性は満たしていると言える.\\
5つの文書の有用性については,これらの文書が実際に検索者の役に立ったかを知る必要があるため,客観的に判断することは困難である.ただし,Apple Inc.に関する文書である4つは有用性を満たす可能性が高く.
中古車買取・中古車査定のアップルに関する文書は有用性を満たす可能性が低い.ただし,検索者が中古車買取・中古車査定をしたいと考えていた場合,結果として検索者の役に立つ情報であるため,有用性を満たす可能性がある.

\end{document}
