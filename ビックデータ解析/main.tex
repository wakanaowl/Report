\documentclass[dvipdfmx,autodetect-engine,titlepage]{jsarticle}
\usepackage[dvipdfm]{graphicx}
\usepackage{ascmac}
\usepackage{fancybox}
\usepackage{listings}
\usepackage{plistings}
\usepackage{itembkbx}
\usepackage{amsmath}
\usepackage{url}
\usepackage{graphics}
\usepackage{listings}
\usepackage{here}

\lstset{%
  language={C},
  basicstyle={\small},%
  identifierstyle={\small},%
  commentstyle={\small\itshape\color[rgb]{0,0.5,0}},%
  keywordstyle={\small\bfseries\color[rgb]{0,0,1}},%
  ndkeywordstyle={\small},%
  stringstyle={\small\ttfamily\color[rgb]{1,0,1}},
  frame={tb},
  breaklines=true,
  columns=[l]{fullflexible},%
  numbers=left,%
  xrightmargin=0zw,%
  xleftmargin=3zw,%
  numberstyle={\scriptsize},%
  stepnumber=1,
  numbersep=1zw,%
  lineskip=-0.5ex%
}

\textheight=23cm
\renewcommand{\figurename}{図}
\renewcommand{\tablename}{表}
\newenvironment{code}
{\vspace{0.5zw}\VerbatimEnvironment  \begin{screen} 
\baselineskip=1.0\normalbaselineskip
 \begin{Verbatim}}
{\end{Verbatim}
\baselineskip=\normalbaselineskip
 \end{screen}\vspace{0.5zw}} 

\title{ビッグデータ解析(A1)\\
第2回レポート\\
固有値と固有ベクトル\\
}
\author{2600200087-2\\Oku Wakana\\奥 若菜}
\date{Oct.24 2022}

\begin{document}

\maketitle

\section{対称行列Aの固有値\begin{math}\alpha ,\beta \end{math}を求めよ}
\begin{eqnarray*}
  A =
\begin{pmatrix}
  1 & 2 \\
  2 & -2 \\
\end{pmatrix}
\end{eqnarray*}

\begin{eqnarray*}
det(A -\lambda E) 
&=& det 
\begin{pmatrix}
  1-\lambda  & 2 \\
  2 & -2-\lambda  \\
\end{pmatrix} \\\\
&=& (1-\lambda )(-2-\lambda )-2 \times 2 \\\\
&=& \lambda ^2 + \lambda  - 6 \\\\
&=& (\lambda -2)(\lambda +3) \\\\
\end{eqnarray*}

\begin{math}
  (\lambda -2)(\lambda +3) = 0
\end{math}
のとき,\begin{math}
  \lambda = 2,-3 
\end{math} より \\

固定値は \begin{math}
  \alpha = 2, \beta = -3
 \end{math} \\


\section{固有値に対応する,大きさ1の固有ベクトルを求めよ}
固有値 \begin{math}
  \alpha = 2
\end{math}について,
\begin{math}
A \overrightarrow{p} = 2 \overrightarrow{p} 
\end{math} に値を当てはめると

\begin{eqnarray*}
  \begin{pmatrix}
    1-\lambda  & 2 \\
    2 & -2-\lambda  \\
  \end{pmatrix} 
  \binom{x}{y} 
  = 2\binom{x}{y} 
\end{eqnarray*}

\begin{eqnarray*}
  x +2y = 2x \\\\
  2x -2y =2y
\end{eqnarray*}

双方の式から,\begin{eqnarray*}
  y= \frac{1}{2} x
\end{eqnarray*}

よって固有値\begin{math}
  \alpha = 2
\end{math}の固有ベクトルは,\begin{math}
  t \neq 0
\end{math}として
\begin{eqnarray*}
  \overrightarrow{p_{\alpha }} = t \binom{2}{1} \\\\
  \overrightarrow{p_{\beta }} = t \binom{\frac{2}{\sqrt{5} } }{\frac{1}{\sqrt{5}} } \\
\end{eqnarray*}


固有値 \begin{math}
  \beta  = -3
\end{math}について,
\begin{math}
A \overrightarrow{p} = -3 \overrightarrow{p} 
\end{math} に値を当てはめると

\begin{eqnarray*}
  \begin{pmatrix}
    1-\lambda  & 2 \\
    2 & -2-\lambda  \\
  \end{pmatrix} 
  \binom{x}{y} 
  = -3\binom{x}{y} 
\end{eqnarray*}

\begin{eqnarray*}
  x +2y = -3x \\\\
  2x -2y =-3y
\end{eqnarray*}

双方の式から,\begin{eqnarray*}
  y= -2 x
\end{eqnarray*}

よって固有値\begin{math}
  \alpha = 2
\end{math}の固有ベクトルは,\begin{math}
  t \neq 0
\end{math}として
\begin{eqnarray*}
  \overrightarrow{p_{\beta }} = t \binom{1} {-2} \\\\
  \overrightarrow{p_{\beta }} = t \binom{\frac{1}{\sqrt{5} } }{\frac{-2}{\sqrt{5}} } \\
\end{eqnarray*}


\section{固有ベクトルが直交していることを示せ}
固有ベクトルの内積を求める
\begin{eqnarray*}
  \overrightarrow{p_{\alpha }} \cdot \overrightarrow{p_{\beta }}
  &=& t \binom{2}{1}\cdot t \binom{1}{-2} \\\\
  &=& t^2 \left\{2 \cdot 1 - 1 \cdot (-2) \right\} \\\\
  &=& 0
\end{eqnarray*}

よって,内積が0であるため直行する

\end{document}
