\documentclass[dvipdfmx,autodetect-engine,titlepage]{jsarticle}
\usepackage[dvipdfm]{graphicx}
\usepackage{ascmac}
\usepackage{fancybox}
\usepackage{listings}
\usepackage{plistings}
\usepackage{itembkbx}
\usepackage{amsmath}
\usepackage{url}
\usepackage{graphics}
\usepackage{listings}

\textheight=23cm
\renewcommand{\figurename}{図}
\renewcommand{\tablename}{表}
\newenvironment{code}
{\vspace{0.5zw}\VerbatimEnvironment  
\begin{screen} 
\baselineskip=1.0\normalbaselineskip
 \begin{Verbatim}}
{\end{Verbatim}
\baselineskip=\normalbaselineskip
 \end{screen}\vspace{0.5zw}} 

\title{情報理工学部 SNコース 3回\\
ワイヤレス通信システム\\
放射電磁界の距離依存性}
\author{26002000872\\Oku Wakana\\奥 若菜}
\date{May 15 2022}

\begin{document}

\maketitle

\section{電気ダイポールによる電磁界}
教科書15ページ図2.2の電気ダイポールの各電磁界成分は、16~17ページの導出により、

\begin{align*}
  H_{\phi } &= \frac{Idl\sin \theta }{4\pi}(\frac{jk}{r}+\frac{1}{r^2})e^{-jkr} \tag*{2・18}
\end{align*}

\begin{align*}
  E_{r} &= \frac{Idl\cos \theta }{j2\pi \omega \epsilon}(\frac{jk}{r^2}+\frac{1}{r^3})e^{-jkr} \\ \tag*{2・19}
\end{align*}

\begin{align*}
  E_{\theta } &= \frac{Idl\sin \theta }{j4\pi \omega \epsilon}(\frac{k^2}{r}-\frac{jk}{r^2}-\frac{1}{r^3})e^{-jkr} \\ \tag*{2・20}
\end{align*}

\begin{align*}
  H_{r} &= H_{\theta} &= E_{\phi } &= 0 \\ \tag*{2・21}
\end{align*}

となり、磁界は \begin{math}\phi \end{math}成分だけを、電界は \begin{math} r,\theta \end{math}の2成分だけを持つこと,\\
電磁界成分は\begin{math}\frac{1}{r^3},\frac{1}{r^2},\frac{1}{r}\end{math}のいずれか、または全ての項から構成されることが分かる.\\
距離依存性について考えると、十分遠方では\begin{math}\frac{1}{r^3},\frac{1}{r^2}\end{math}の項による影響は受けないため、

\begin{align*}
  &= \frac{Idl \mu_0 \sin \theta }{j2\lambda r \sqrt{\epsilon _{0}\mu _0 }}e^{-jkr}\\
\end{align*}

\end{document}