\documentclass[dvipdfmx,autodetect-engine]{jsarticle}
\usepackage[dvipdfm]{graphicx}
\usepackage{ascmac}
\usepackage{fancybox}
\usepackage{listings}
\usepackage{plistings}
\usepackage{itembkbx}
\usepackage{amsmath}
\usepackage{url}
\usepackage{graphics}
\usepackage{listings}
\usepackage{here}

\lstset{
  basicstyle={\ttfamily},
  identifierstyle={\small},
  commentstyle={\smallitshape},
  keywordstyle={\small\bfseries},
  ndkeywordstyle={\small},
  stringstyle={\small\ttfamily},
  frame={tb},
  breaklines=true,
  columns=[l]{fullflexible},
  numbers=left,
  xrightmargin=0zw,
  xleftmargin=3zw,
  numberstyle={\scriptsize},
  stepnumber=1,
  numbersep=1zw,
  lineskip=-0.5ex
}

\textheight=23cm
\renewcommand{\figurename}{図}
\renewcommand{\tablename}{表}
\newenvironment{code}
{\vspace{0.5zw}\VerbatimEnvironment
\begin{screen} 
\baselineskip=1.0\normalbaselineskip
 \begin{Verbatim}}
{\end{Verbatim}
\baselineskip=\normalbaselineskip
 \end{screen}\vspace{0.5zw}} 

 \title{応募履歴書} 
 \author{\empty}
 \date{\empty}
 \begin{document} 
 \maketitle

\subsubsection*{名前}
奥 若菜  /  Oku Wakana
\subsubsection*{住所}
603-8425  京都府京都市北区紫竹下緑町88-1エスパス北山101号室

\subsubsection*{電話番号}
080-4258-2702

\subsubsection*{メールアドレス}
wakanappa22@gmail.com

\subsubsection*{希望職種}
ソリューションアーキテクト

\subsubsection*{大学名}
立命館大学

\subsubsection*{学部 / 学科}
情報理工学部 / 情報理工学科

\subsubsection*{卒業予定学位}
学士

\subsubsection*{卒業年月}
2024年3月 卒業予定

\subsubsection*{研究室}
2022年9月 配属予定

\subsubsection*{サークル}
watnow (学生IT団体)

\subsubsection*{TOEICスコア}
635点

\subsubsection*{使用可能なプログラミング言語}
C : 2年半 / Go: 1年 / Python : 半年 / Swift : 半年
\subsubsection*{保有するITスキル}
マルウェア解析 / サイバー攻撃防御 / データベース構築 / iOSアプリ開発 / Webアプリ開発
\subsubsection*{インターンシップ経験}
クックパッド1dayインターンシップ (2021年2月25日)
\subsubsection*{アルバイト経験}
NPO法人スーパーサイエンスキッズの「プログラミングワークショップ」にアシスタントとして参加し、子供たちの質問に答えたり、一緒にプログラミングをしています。
\subsubsection*{募集を知ったきっかけ}
type就活「エンジニア学生のためのIT業界研究セミナー」 (2022年6月11日)
\subsubsection*{専攻内容について}
専攻しているセキュリティ・ネットワークコースで、ワイヤレス通信システムや暗号・セキュリティ技術について学んでいます。
演習では、TCP通信を行うWebサーバの構築、IoTデバイスの開発、脆弱性への攻撃などを行うことで、それらの原理を理解しつつ、実践的な技術を習得しています。
\subsubsection*{最近気になるテクノロジー}
公開鍵暗号の一つである楕円曲線暗号に注目しています。楕円曲線暗号は、現在広く用いられているRSA暗号と比較して、1/10以下の鍵サイズで同じ安全性を実現することができます。IoT機器が普及し、ますますコンパクトな暗号機能の装備が必要になる
中で、楕円曲線暗号の役割は重要になっていくと考えています。
\subsubsection*{将来キャリアで実現したいこと}
将来は、様々な情報サービスの基盤となる、セキュアな情報インフラの構築に貢献したいです。社会のどの分野においても、情報が利用されるなかで、信頼されるサービスへの需要は急激に増加していくと考えております。
実世界にあらゆる方面から影響を与えうる情報サービスを、セキュリティやインフラ等の土台から支え、社会に安全・安心を与える役割をしたいです。
\end{document}