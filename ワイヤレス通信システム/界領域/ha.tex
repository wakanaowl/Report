\documentclass{article}
\usepackage[utf8]{inputenc}
\usepackage{amsmath}

\title{ワイヤレス通信システム第4回レポート}
\author{26002002549 塚本覇虎}
\date{2022/5/29}

\begin{document}

\maketitle
自由空間において、
\begin{eqnarray*}
B &=& \nabla \times A
\end{eqnarray*}
は、ファラデーの法則から、
\begin{eqnarray*}
\nabla \times E &=& -\frac{\partial}{\partial t}(\nabla \times A)\\
 \nabla \times (E+\frac{\partial A}{\partial t})&=& 0
\end{eqnarray*}
$\nabla \times \nabla \phi = 0$より
\begin{eqnarray*}
E+\frac{\partial A}{\partial t}&=&\nabla \phi\\
E &=& \nabla \phi-j \omega A
\end{eqnarray*}
アンペア・マクスウェルの法則とベクトル恒等式から
\begin{eqnarray*}
\nabla \times H &=& J + \frac{\partial D}{\partial t}\\
\nabla \times \frac{B}{\mu} &=& J + j \omega \epsilon\\
\frac{1}{\mu}(\nabla\nabla\cdot A - \nabla^2 A)&=&J + j \omega \epsilon
\end{eqnarray*}
より、
\begin{eqnarray*}
\frac{1}{\mu}(\nabla\nabla\cdot A - \nabla^2 A)&=&J + j \omega \epsilon(\nabla\phi - j\omega A)\\
\nabla^2 A-\nabla(\nabla\cdot A -j \omega \epsilon \mu \phi ) + \omega^2 \epsilon \mu A &=& -\mu J
\end{eqnarray*}
$k^2 = \omega^2 \epsilon \mu$とすると、
\begin{eqnarray*}
(\nabla^2 + k^2)A_z &=& -\mu J_z
\end{eqnarray*}
これをラプシアンの球座標表現を用い展開すると、原点以外でのベクトルポテンシャルは
\begin{eqnarray*}
\frac{1}{r^2}\frac{\partial}{\partial r}r^2\frac{\partial}{A_z}+k^2 A_z &=& 0
\end{eqnarray*}
を満足する。この波動方程式を満足する球波面の基本解のうち、外向波のベクトルポテンシャルは、$C_1$を比例定数として、
\begin{eqnarray*}
A_z&=&C_1\frac{e^{-jkr}}{r}
\end{eqnarray*}
と表現できる。
\begin{eqnarray*}(\nabla^2 + k^2)A_z &=& -\mu J_z\end{eqnarray*}の両辺を$_0$の微小体積で積分し、ガウスの定理を適用すると、
\begin{eqnarray*}
\int_V \nabla^2 A_z dV + k^2 \int_V A_z dV &=& -\mu \int_V J_z dV\\
\int_V\nabla\cdot\nabla A_z dV &=& -k^2 \int_V A_z dV - \mu \int_V J_z dV\\
\oint_S \nabla A_z dS &=& -k^2 \int_V A_z dV - \mu \int_V J_z dV\\
\int_0^{2\pi} \int_0^\pi \nabla A_z \cdot \hat{r} {r_0}^2\sin{\theta}d\theta d\phi &=& -k^2 \int_V A_z dV - \mu \int_V J_z dV
\end{eqnarray*}
電流密度の体積積分は$Idl$に収束するため、
\begin{eqnarray*}
4\pi C_1 &=& \mu Idl\\
C_1 &=& \frac{\mu Idl}{4\pi}
\end{eqnarray*}
よって、
\begin{eqnarray*}
A_z &=& \mu Idl \frac{e^{-jkr}}{4\pi r}
\end{eqnarray*}
である。この式を
\begin{eqnarray*}
H &=& \frac{1}{\mu}\nabla \times A\\
E &=& -j\omega A + \frac{1}{j\omega \mu \epsilon}\nabla\nabla\cdot A
\end{eqnarray*}
に適用する。
まず、$z$成分方向だけをもつベクトルポテンシャルを球座標系で表すと、
\begin{eqnarray*}
A&=& \mu Idl \frac{e^{-jkr}}{4\pi r }(\hat{r}\cos\theta - \hat{\theta}\sin\theta)
\end{eqnarray*}
これより、
\begin{eqnarray*}
H &=& \frac{1}{\mu}\nabla\times [\mu Idl \frac{e^{-jkr}}{4\pi r }(\hat{r}\cos\theta - \hat{\theta}\sin\theta)]\\
&=&\frac{Idl \sin\theta}{4\pi}(\frac{jk}{r}+ \frac{1}{r^2})e^{-jkr}
\end{eqnarray*}
であるから、
\begin{eqnarray*}
H_{\phi}&=& \frac{Idl \sin\theta}{4\pi}(\frac{jk}{r}+ \frac{1}{r^2})e^{-jkr}
\end{eqnarray*}
電磁界の双対性から、
\begin{eqnarray*}
E_{\phi}&=& -\frac{j\omega \mu I \pi a^2 \sin\theta}{4\pi}(\frac{jk}{r} + \frac{1}{r^2})e^{-jkr}
\end{eqnarray*}
以上によって導出できた。
\end{document}