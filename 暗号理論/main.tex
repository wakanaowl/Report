\documentclass[dvipdfmx,autodetect-engine,titlepage]{jsarticle}
\usepackage[dvipdfm]{graphicx}
\usepackage{ascmac}
\usepackage{fancybox}
\usepackage{listings}
\usepackage{plistings}
\usepackage{itembkbx}
\usepackage{amsmath}
\usepackage{url}
\usepackage{graphics}
\usepackage{listings}
\usepackage{here}

\lstset{%
  language={C},
  basicstyle={\small},%
  identifierstyle={\small},%
  commentstyle={\small\itshape\color[rgb]{0,0.5,0}},%
  keywordstyle={\small\bfseries\color[rgb]{0,0,1}},%
  ndkeywordstyle={\small},%
  stringstyle={\small\ttfamily\color[rgb]{1,0,1}},
  frame={tb},
  breaklines=true,
  columns=[l]{fullflexible},%
  numbers=left,%
  xrightmargin=0zw,%
  xleftmargin=3zw,%
  numberstyle={\scriptsize},%
  stepnumber=1,
  numbersep=1zw,%
  lineskip=-0.5ex%
}

\textheight=23cm
\renewcommand{\figurename}{図}
\renewcommand{\tablename}{表}
\newenvironment{code}
{\vspace{0.5zw}\VerbatimEnvironment  \begin{screen} 
\baselineskip=1.0\normalbaselineskip
 \begin{Verbatim}}
{\end{Verbatim}
\baselineskip=\normalbaselineskip
 \end{screen}\vspace{0.5zw}} 

\title{暗号理論\\
}
\author{2600200087-2\\Oku Wakana\\奥 若菜}
\date{Jun.5 2022}

\begin{document}

\maketitle

\section{AES(Advanced Encryption Standard)}
\subsection{概要}
AES(Advanced Encryption Standard)は、2001年にアメリカが標準規格として定めた共通鍵暗号アルゴリズムである。
AES以前には、IBM社が構築したDESという共通鍵暗号アルゴリズムが標準だったが、より強力なプロセッサの登場によりセキュリティ上の懸念が浮上した。
そこで、アメリカ国立標準技術研究所(NIST)が公募し、ベルギーの研究者Joan DaemenとVincent Rijimenが設計した
RijndaelというアルゴリズムがAESとして採用された。

\subsection{AESの暗号化操作}
比較的短い56bitの鍵を使用するDESとは異なり、AESは128bit、192bit、256bitのいずれかの鍵長を使用してデータを暗号化と復号化を行う。
AESは平文(入力データ)を128bit単位のブロックに分割し、
\section{TLS}
\subsection{概要}
同じく問4について、動作周波数が800MHzと1.2GHzである二つの送信アンテナがあり、いずれも最大開口形は1.2mである。これらのアンテナの指向性を測定するために必要な距離を求めよ。なお、測定に用いる受信アンテナは波長に比べて小さいとする。

\subsection{解答}
アンテナから観測点までの距離を\begin{math}r\end{math}とすると、\begin{math}
  rが r_f=2D^2/ \lambda 
\end{math}より大きいとき、アンテナの放射強度の方向依存性を表す放射パターンが距離により変化しないとみなせる。よって、今回は\begin{math}
  r_f
\end{math}をアンテナの指向性を測定するために必要な最小の距離とする。\\

(1) 800MHzのとき
\begin{eqnarray*}
  \lambda = \frac{3.0 \times  10^8 [m/s]}{8.0 \times  10^8 [Hz]}
  = 0.375 [m]
\end{eqnarray*}

\begin{eqnarray*}
  r_f = \frac{2(1.2)^2[m^2]}{0.375[m]}
  = 7.68[m]\\
\end{eqnarray*}

(2)1.2GHzのとき
\begin{eqnarray*}
  \lambda = \frac{3.0 \times  10^8 [m/s]}{1.2 \times 10^10[Hz]}
  =0.025[m]
\end{eqnarray*}

\begin{eqnarray*}
  r_f = \frac{2(1.2)^2[m^2]}{0.025[m]}
  =115.2[m]\\\\
\end{eqnarray*}

\section{微小ダイポールアンテナの指向性利得}
\subsection{問題}
同じく問5について、原点z軸に沿って配置した微小ダイポールアンテナの指向性利得\begin{math}
  G_dをc
\end{math}の関数として導出する。

\subsection{解答}
教科書(2・23)より指向性利得は、以下の式で求められる。

\begin{eqnarray*}
  G_d(\theta) = \frac{{| D(\theta) \vert }^2}{\frac{1}{4\pi} \int_{0}^{2\pi}  \,d\phi \int_{0}^{\pi} {| D(\theta) \vert }^2 \sin\theta \,d\theta } \\\\
\end{eqnarray*}

微小ダイポールアンテナの指向性係数は、
\begin{eqnarray*}
  D(\theta) = \sin\theta \\
\end{eqnarray*}

よって、微小ダイポールアンテナの指向性利得\begin{math}
  G_d(\theta)
\end{math}は

\begin{eqnarray*}
  G_d(\theta) &=& \frac{\sin^2\theta }{\frac{1}{4\pi} \int_{0}^{2\pi} \,d\phi \int_{0}^{\pi} \sin^3\theta \,d\theta  }  \\\\\\
  &=& \frac{4\pi \sin^2\theta }{ \int_{0}^{2\pi} \,d\phi \frac{1}{2} \int_{0}^{\pi}  \sin\theta - \sin\theta\cos2\theta \,d\theta  } \\\\\\
  &=& \frac{4\pi \sin^2\theta }{ \int_{0}^{2\pi} \,d\phi \frac{1}{4} \int_{0}^{\pi}  -3\sin\theta - \sin3\theta\,d\theta  } \\\\\\
  &=& \frac{16\pi \sin^2\theta }{ 2\pi [-3\cos\theta + \frac{1}{3}\cos3\theta]_{0}^{\pi}} \\\\\\
  &=& \frac{16\pi \sin^2\theta }{ 2\pi \frac{16}{3}} \\\\\\
  &=& \frac{3\sin^2\theta}{2}\\\\
\end{eqnarray*}

と求められた。

\end{document}
